\section{Conclusions}
The LBNF is the long baseline neutrino oscillations experiment under development which will be hosted by two large physics laboratories in USA: Fermilab in Illinois and SURF in South Dakota. The collaboration already include $>750$ people and many of them had experience in neutrino physics with other experiments. The first collaboration meeting took place on April 16th-18th of 2015 in Fermilab, $\sim 200$ scientists came together to discuss their progress and plans towards the LBNF experiment future operation. Completing CDR is one of the short-term goals and the document is well-progressing and it's drafts are partially availabe at the LBNF website. The far detector installation is planned in 2021-2022 in the cavern of the former Homestake mine which in the past hosted another neutrino experiment - the Homestake experiment in 1968 which was the first one to claim the solar neutrino problem. The near detector in Fermilab will require a cavern excavation 60 meters underground and a building construction above it, on surface. The neutrino beam production system will be performed by already existing Fermilab accelerator complex, by team which already has experience in such work: the MINOS experiment which was operating 2005-2012 and is currently under upgrade.    \\  

The LBNF's baseline of 1300 km, expected beam power of 2 MW, 40 kt of liquid argon far detector and strong team of people with experience in other experiments of this kind, makes the LBNF the most ambitious neutrino oscillations facility ever created. In addition to presicion measurements of such neutrino mixing parameters as $\theta_{12}$, $\theta_{23}$, $\theta_{13}$, $|\Delta{m_{12}}^2|$, $|\Delta{m_{31}}^2|$, it's expected to have enough sensitivity to determine the neutrino mass hierarchy and the CP-violation phase $\delta_{CP}$ which were never determined before.\\

However, despite all the advantages of the LBNF and expectations the scientific society has to it, there is still something which this experiment will not be able to measure. For example, the neutrino masses themselves - because neutrino oscillations are only sensitive to differences. Neutrino mass measurement require different kind of experiments - for instance, studiyng high energy cut-off on the electron energy spectrum in beta-decay of tritium. By this time, all the experiments trying to measure it were able only to set upper limits on neutrino masses (\cite{ref_Griffiths}, 11.4).  
