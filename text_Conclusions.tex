\section{Conclusions}
Neutrino oscillations were first observed in the 1960s and confirmed in 1998. Neutrinos were believed to be massless for a long time, but the oscillations assume neutrinos to be massive. This phenomenon made the scientific community rethink the theory of weak interactions. While several possible theoretical frameworks to incorporate neutrino masses into the Standard Model have been suggested, it is not clear which hypothesis is true. Precision measurement of neutrino oscillation parameters may bring insight into the understanding of matter-antimatter asymmetry in the Universe and general picture of the generations of the fundamental Standard Model particles. \\ \\
Neutrino oscillations can be described with four parameters of the neutrino mixing matrix (three angles $\theta_{12}$, $\theta_{23}$, $\theta_{13}$ and complex CP-violating phase $\delta$) and the two mass differences (${\Delta}m_{21}$ and ${\Delta}m_{32}$).\\ \\
Neutrino oscillations can be studied with solar, atmospheric, reactor, or accelerator neutrino experiments. The key idea of any neutrino oscillation experiment is to observe and quantitatively describe disappearance of neutrinos/antineutrinos which were produced and/or appearance of neutrinos/antineutrinos which were not produced. Then probability of neutrinos to oscillate is built and fitted with the assumptions of a certain model, and the neutrino oscillation parameters are extracted.\\ \\
Significant progress was made in recent years in neutrino oscillation physics with all four types of experiments. Three angles $\theta_{12}$, $\theta_{23}$, $\theta_{13}$ have been measured with several degrees precision, absolute values of ${\Delta}m_{21}$ and ${\Delta}m_{32}$ were measured too, and it is known that $m_2 > m_1$. However, all previous experiments were not sensitive enough to determine the sign of ${\Delta}m_{32}$ and the phase $\delta$.\\ \\
That is why the new LBNF/DUNE experiment was proposed. It will be an accelerator long baseline neutrino oscillations experiment. The highest ever neutrino beam power, the longest baseline and the most effective far detector would make LBNF/DUNE the most ambitious neutrino oscillations experiment in the world and sensitive to effects which were not observed by any other experiment to date. LBNF/DUNE is expected to determine the sign of ${\Delta}m_{32}$ with high significance, phase $\delta$ in broad range of values, and provide the most precise test of the three-neutrino oscillations theory.\\ \\
