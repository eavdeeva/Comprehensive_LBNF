\section{Neutrino Oscillations. History and Status}
%\subsection{First Discovery and Confirmation}

The history of the discovery of neutrino oscillations is described in \cite{ref_Griffiths}, chapter 11. The first evidence of neutrino oscillations was seen in the Homestake experiment in 1968 with solar neutrinos. This experiment used a Chlorino radiochemical detector. Neutrinos interacted with chlorine-37 atoms and converted them to argon-37 through the reaction $\nu_e+^{37}Cl \rightarrow ^{37}Ar+e$ or, at the more fundamental level, $\nu_e+n \rightarrow p+e$. The argon atoms were then separated and counted. The experiment registered a number of neutrinos three times smaller than theoretically predicted. The phenomenon was called the ``solar neutrino problem".  The detector was sensitive to electron neutrinos only. Soon after Bruno Pontecorvo proposed the explanation to the solar neutrino problem that neutrino can change its flavor on its way from the Sun to the detector. The theory was confirmed  by Super-Kamiokande and Sudbury Neutrino Observatory (SNO) collaborations. \\ \\
The Super-Kamiokande experiment used a water detector and could register any sort of neutrino through $e+\nu \rightarrow e+\nu$ scattering. However the NC scattering can not distinguish between different neutrino flavors. Also, electron neutrinos could interact through CC, which made the detection efficiency of electron neutrinos 6.5 times higher than other flavors (the left Feynman diagram in Fig. \ref{fig:NuScattering} is possible for any neutrino flavor but the middle diagram is possible only for electron neutrino). Therefore, Super-Kamiokande was able to register any neutrino but could not distinguish between neutrino flavor and had lower detection efficiency for non-electron neutrinos. They assumed all neutrinos to be electron neutrinos and recorded $45\%$ of the predicted amount. Then the SNO, which used heavy water and was able to distinguish the electron neutrino flux from the total neutrino flux, confirmed that some of the neutrinos coming from the Sun are detected as $\nu_\mu$ or $\nu_\tau$. The reactions in the working volumes of the three detectors can be summarized as the following:\\ 
\begin{itemize}
\item Homestake experiment (1968): $\nu_e + ^{37}Cl \rightarrow ^{37}Ar+e$
\item Super-Kamiokande experiment (1998): $\nu + e \rightarrow \nu + e$ 
\item Solar neutrino observatory (2002): $\nu_e + d \rightarrow p+p+e$, $\nu+d \rightarrow n+p+\nu$, $\nu+e \rightarrow \nu+e$
\end{itemize}
The SNO reported $\nu_e$ flux to be $35\%$ of the predicted flux. Comparing it to the Super-Kamiokande results and knowing that Super-Kamiokande was 6.5 times less sensitive to $\nu_\mu$ and $\nu_\tau$, one obtains:\\ \\
$N_{SNO}=0.35 \cdot N_{th}$,\\ \\
$N_{SK}^{CORR1}=0.45 \cdot N_{th}=0.35 \cdot N_{th}+0.1 \cdot N_{th}$,\\ \\
$N_{SK}^{CORR1}=\frac{N_{SK}^{REG}}{\epsilon^{e}}=0.45 \cdot N_{th} $,\\ \\
$N_{SK}^{CORR2}=\alpha \cdot \frac{N_{SK}^{REG}}{\epsilon^{e}}+(1-\alpha) \cdot \frac{N_{SK}^{REG}}{\epsilon^{\mu/\tau}}=\alpha \cdot \frac{N_{SK}^{REG}}{\epsilon^{e}}+(1-\alpha) \cdot \frac{N_{SK}^{REG}}{\epsilon^{e}/6.5}$,\\ \\
$\alpha=0.35/0.45$,\\ \\
$N_{SK}^{CORR2}=0.35 \cdot N_{th}+0.65 \cdot N_{th}=N_{th}$,\\ \\
where $N_{th}$ is a number of theoretically predicted neutrinos, $N_{SNO}$ is a number of neutrinos registered by the SNO experiment corrected by the neutrino detection efficiency, $N_{SK}^{REG}$ is a number of neutrinos registered by Super-Kamiokande, $N_{SK}^{CORR1}$ is a number of neutrinos registered by Super-Kamiokande corrected by the electron detection efficiency, $\epsilon^{e}$ is the electron detection efficiency, $N_{SK}^{CORR2}$ is a number of neutrinos registered by Super-Kamiokande corrected in assumption that part of registered neutrinos were $\nu_e$ and the other part were $\nu_\mu$ or $\nu_\tau$. This result confirmed the neutrino oscillations theory and resolved the solar neutrino problem.\\ \\
%\subsection{First measurements of the neutrino oscillation parameters}
There are four types of experiments which allow to study neutrino properties: solar, atmospheric, reactor, and accelerator neutrino experiments \cite{ref_PDG}. The idea of any neutrino oscillation measurement is to observe and quantify disappearance of $\nu_f$($\bar{\nu_f}$) and/or appearance of $\nu_{f'}$($\bar{\nu_{f'}}$), where f is a flavor of neutrinos produced in the reaction in the Sun, atmosphere, reactor or target, and $f'$ is the flavor of neutrinos which were not produced in the reaction.\\ \\
The Sun is a source of electron neutrinos. They are produced in the reaction $4p \rightarrow {^4}He+2e^{+}+2\nu_{e}$. Solar neutrino experiments (as the three described above) focus on the disappearance of $\nu_e$ and appearance of $\nu_\mu$ or $\nu_\tau$. These experiments are characterized by very long baselines (distance from the Sun to the Earth, $\simeq 1.5 \cdot 10^8$ km) and neutrino energies of $\sim 1$ MeV. Solar neutrino experiments are sensitive to ${\Delta}m^2_{21}={\Delta}m^2_{Sol}$, which is also called Solar mass splitting. Matter effects in the Sun makes such experiments sensitive to the sign of ${\Delta}m^2_{21}$, and it was determined that $m2>m1$ \cite{ref_presentation_MH}. The Solar neutrino experiments also provided the first measurements of the mixing angle $\theta_{12}$.\\ \\
The nuclear reactors produce $\bar{\nu_e}$s in beta decay of radioactive elements as $n \rightarrow p + e^- + \bar{\nu_e}$, and then antineutrinos further oscillate and are detected. Typical baselines of such experiments vary from 1 to 100 km, and the energies of emitted antineutrinos are $\sim 1$ MeV. Reactor experiments provide measurements of ${\Delta}m^2_{21}$, $\theta_{12}$ and $\theta_{13}$.\\ \\
The source of atmospheric neutrinos is cosmic rays scattering from air molecules as shown in Fig. \ref{fig:cosmicMuons}. Pions produced in the air decay as $\pi^{\pm} \rightarrow \mu^{\pm}+\nu_\mu(\bar{\nu_\mu})$ with further muon decay as $\mu^{\pm} \rightarrow e^{\pm}+\nu_e(\bar{\nu_e})+\bar{\nu_\mu}(\nu_\mu)$, which lead to a flux ratio $\Phi(\nu_\mu+\bar{\nu_\mu}):\Phi(\nu_e+\bar{\nu_e}) \approx 2:1$. Baselines of atmospheric neutrinos experiments are $\sim 10^4$ km, and neutrino energies are $\sim 1$ GeV. Atmospheric neutrino experiments are more sensitive to ${\Delta}m^2_{32}={\Delta}m^2_{Atm}$, which is also called atmospheric mass splitting, and also provide measurements of $\theta_{23}$. Such experiments are also potentially capable of measuring CP-violating phase $\delta$, but experiments has been performed so far were not sensitive enough to measure $\delta$. \\ \\
Accelerators can produce $\nu_\mu$ or $\bar{\nu_\mu}$ similarly as $\nu_\mu$($\bar{\nu_\mu}$) are produced in the atmosphere but accelerators can produce high purity $\nu_\mu$ or $\bar{\nu_\mu}$ beam by choice. Neutrino energies produced by accelerators are $\sim 1$ GeV, and the baselines are a few hundred kilometers. Fixed baselines and better understood neutrino beam spectra lead to potentially more precise measurements than atmospheric experiments. Accelerator experiments are measuring all neutrino oscillation parameters but the most important targets are ${\Delta}m^2_{32}$, $\theta_{23}$, $\theta_{13}$ and $\delta$. However, similarly to atmospheric experiments, the accelerator experiments performed so far were not sensitive enough to measure $\delta$. Having neutrinos long distance to travel through matter make accelerator experiments potentially sensitive to mass hierarchy. Now it is only known that $m_2>m_1$ and $|{\Delta}m_{21}| \ll |{\Delta}m_{32}|$ but it is not known whether $m_3>m_2>m_1$ or $m_2>m_1>m_3$.\\ \\
%\subsection{Recent Experimental Results}
To summarize, among the neutrino oscillation parameters all three mixing angles and two mass differences have been measured, however the sign of ${\Delta}m^2_{32}$ is not known and CP-violating phase $\delta$ is also not known. Currently available experimental results are summarized in the Table \ref{tab:MeasuredPars}.\\ \\
\begin{table}[h]
  \begin{center}
  \caption{ Neutrino oscillation parameters measured in other experiments \cite{ref_PDG}.}
  \begin{tabular}{|c|c|c|c|}
     Parameter & Value and uncertainty & Comment \\ \hline
     $sin^2(2\theta_{12})$ &  $0.846\pm0.021$ & \\ \hline 
     $sin^2(2\theta_{23})$ &  $0.999${\tiny{$^{+0.001}_{-0.018}$}} & if normal mass hierarchy \\ \hline 
     $sin^2(2\theta_{23})$ &  $1.000${\tiny{$^{+0.000}_{-0.017}$}}  & if inverted mass hierarchy \\ \hline 
     $sin^2(\theta_{13}), 10^{-2}$ &  $9.3\pm0.8$  & first measured in 2012\\ \hline 
     ${\Delta}m^2_{21}$, $10^{-5} eV^2$ &  $7.53\pm0.18$  &  $m_{2}>m_{1}$   \\ \hline 
     ${\Delta}m^2_{32}, 10^{-3} eV^2$ &  $2.44\pm0.06$  &  if normal mass hierarchy     \\ \hline
     ${\Delta}m^2_{32}, 10^{-3} eV^2$ &  $2.52\pm0.07$  &  if inverted mass hierarchy     \\ \hline
     $m_\nu, eV$ &  $<2$  &      \\ \hline 
  \end{tabular}
  \label{tab:MeasuredPars}
  \end{center}
\end{table}
%Section 14.5 in [REFERENCE] describes measurements of $\Delta{m^2_A}$ and $\theta_A$, splitting between atmospheric neutrino and accelerator experiments results. Section 14.6 reviews measurements of $\theta_{13}$ which was measured recently.
The following questions related to neutrino oscillations remain unknown: 
\begin{itemize}
  \item are the massive neutrinos Dirac or Majorana?
  \item what is the mass hierarchy ($m_3>m_2>m_1$ or $m_2>m_1>m_3$)?
  \item what are the absolute values of the neutrino masses?
  \item how does the CP-symmetry behave in the lepton sector (what is the value of $\delta$ in the neutrino mixing matrix)?
  \item are the neutrino oscillations an indication of a new fundamental symmetry in particle physics?
  \item what is the relation between neutrino and quark mixing (if any)?
  \item can better understanding of neutrino mixing give a hint to matter-antimatter asymmetry in the Universe?
  \item what is the octant of $\theta_{23}$ angle?
\end{itemize} 
