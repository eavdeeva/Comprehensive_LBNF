\section{Neutrino Oscillations. History and Status}
\subsection{First Discovery and Confirmation}
History of the neutrino oscillations discovery is described in \cite{ref_Griffiths}, chapter 11. The first evidence of the neutrino oscillations had place in the Homestake experiment in 1968 with solar neutrinos which registered number of neutrinos three times smaller then theoretically predicted. The phenomenon was called "solar neutrino problem". This experiment used Chlorino radiochemical detector. Neutrino interacted with chlorine-37 atom and converted it to argon-37 through the reaction $\nu_e+^{37}Cl \rightarrow ^{37}Ar+e$ or, at more fundamental level, $\nu_e+n \rightarrow p+e$. Then argon atoms were separated and counted. The detector was sensitive to electron neutrinos only. Soon after Bruno Pontecorvo proposed the explanation to the solar neutrino problem that neutrino can change it's flavor on it's way from the Sun to the detector. The theory was confirmed  by Super-Kamiokande and Sudbury Neutrino Observatory (SNO) collaborations. This experiment used water detector and could register any sort of neutrino through the $e+\nu \rightarrow e+\nu$ scattering. But the NC scattering can not distinguish between different neutrino flavors and also electron neutrinos could interact through CC which made detection efficiency of electron neutrinos 6.5 times higher than other flavors (the left Feynmann diagram at fig. \ref{fig:NuScattering} is possible for any neutrino flavor but the middle and right diagrams - only for electron neutrino). Thus, the super -Kamiokande were able to register any neutrino but couldn't distinguish between neutrino flavor and had lower detection efficiency for non-electron neutrinos. They assumed all neutrinos to be electron neutrinos and recorded $45\%$ of the predicted amount. Then the SNO which used heavy water and were able to measure separately electron and total neutrino flux, confirmed that some of neutrinos coming from the Sun are registered as $\nu_\mu$ and $\nu_\tau$. The reactions in the working volumes of the three detectors can be summarized as the following:\\
\begin{itemize}
\item Homestake experiment (1968): $\nu_e + ^{37}Cl \rightarrow ^{37}Ar+e$
\item Super-Kamiokande experiment (1998): $\nu + e \rightarrow \nu + e$ 
\item Solar neutrino observatory (2002): $\nu_e + d \rightarrow p+p+e$, $\nu+d \rightarrow n+p+\nu$, $\nu+e \rightarrow \nu+e$
\end{itemize}
The SNO reported $\nu_e$ flux to be $35\%$ of the predicted flux. Comparing it to the Super-Kamiokande results and knowing that Super-Kamiokande was 6.5 times less sensitive to $\nu_\mu$ and $\nu_\tau$, one could get:\\
$N_{SNO}=0.35 \cdot N_{th}$\\
$N_{SK}^{CORR1}=0.45 \cdot N_{th}=0.35 \cdot N_{th}+0.1 \cdot N_{th}$\\
$N_{SK}^{CORR1}=\frac{N_{SK}^{REG}}{\epsilon^{e}}=0.45 \cdot N_{th} $\\
$N_{SK}^{CORR2}=\alpha \cdot \frac{N_{SK}^{REG}}{\epsilon^{e}}+(1-\alpha) \cdot \frac{N_{SK}^{REG}}{\epsilon^{\mu/\tau}}=\alpha \cdot \frac{N_{SK}^{REG}}{\epsilon^{e}}+(1-\alpha) \cdot \frac{N_{SK}^{REG}}{\epsilon^{e}/6.5}$\\
$\alpha=0.35/0.45$\\
$N_{SK}^{CORR2}=0.35 \cdot N_{th}+0.65 \cdot N_{th}=N_{th}$\\

After that, the neutrino oscillations theory is considered to be confirmed and the solar neutrino problem - resolved. 


\subsection{First measurements of the neutrino oscillation parameters}

{\textcolor{red}{angles, mass differences, delta m12, theta13, why delta m31 was not measured and why deltaCP was not measured; read PDG chapter probably }} 

\subsection{Recent Experimental Results}
The neutrino oscillation parameters measured in other experiments are summarized in the table \ref{tab:MeasuredPars} as quoted in the PDG \cite{ref_PDG} (section Particle Listings $\rightarrow$ Leptons $\rightarrow$ Neutrino Mixing):\\
\begin{table}[h]
  \begin{center}
  \caption{ Neutrino oscillation parameters measured in other experiments}
  \begin{tabular}{|c|c|c|c|}
     Parameter & Value and uncerntainty & Comment \\ \hline
     $sin^2(2\theta_{12})$ &  $0.846\pm0.021$ & \\ \hline 
     $sin^2(2\theta_{23})$ &  $0.999^{+0.001}_{-0.018}$ & if normal mass hierarchy \\ \hline 
     $sin^2(2\theta_{23})$ &  $1.000^{+0.000}_{-0.017}$  & if inverted mass hierarchy \\ \hline 
     $sin^2(\theta_{13}), 10^{-2}$ &  $9.3\pm0.8$  & only measured in 2012\\ \hline 
     ${\Delta}m^2_{21}$, $10^{-5} eV^2$ &  $7.53\pm0.18$  &  $m_{2}>m_{1}$   \\ \hline 
     ${\Delta}m^2_{32}, 10^{-3} eV^2$ &  $2.44\pm0.06$  &  if normal mass hierarchy     \\ \hline
     ${\Delta}m^2_{32}, 10^{-3} eV^2$ &  $2.52\pm0.07$  &  if inverted mass hierarchy     \\ \hline 
  \end{tabular}
  \label{tab:MeasuredPars}
  \end{center}
\end{table}

%Section 14.5 in [REFERENCE] describes measurements of $\Delta{m^2_A}$ and $\theta_A$, splitting between atmospheric neutrino and accelerator experiments results. Section 14.6 reviews measurements of $\theta_{13}$ which was measured recently.

According to Particle Data Group Review \cite{ref_PDG} the following questions will be the main priority to answer by current and future neutrino experiments:
\begin{itemize}
  \item whether the massive neutrinos are Dirac or Majorana (Dirac means neutrinos and antineutrinos are dirrefent particles; Majorana means neutrinos are their own's antineutrinos)
  \item what is the mass hierarchy
  \item what the absolute values of neutrino masses are
  \item how does the CP-symmetry behaves in the lepton sector
  \item are the neutrino oscillations indication of new fundamental symmetry in particle physics
  \item what is the relation between neutrino and quark mixing if any
  \item what is the nature of the CP-violation terms in the neutrino mixing matrix
  \item can better understanding of neutrino mixing give a hint to baryon assymetry in the Universe 
\end{itemize} 
In addition, more precise measurement of already measured mixing matrix parameters $\theta_{12}$, $\theta_{23}$, $\theta_{13}$, $|\Delta{m_{12}}^2|$, $|\Delta{m_{31}}^2|$ is also prioritized part of new neutrino experiments physics programs.
