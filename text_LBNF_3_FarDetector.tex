\subsection{Far Detector}

Far detector measures the neutrino flux and energies. After travelling 1300 km, neutrino beam becomes more scattered and therefore far detector needs larger volume to detect neutrinos. It only register electron neutrinos and doesn't aim on as high precision as near detector. That is why far detector is so dramatically different from near detector.\\
LBNF/DUNE far detector will be located at SURF in South Dakota. There will be four modules, 10,000 tonnes of liquid argon each, placed into four caverns 1500 m underground. Each module will be 15 m wide, 12 m high and 58 m long, along the beam direction. The caverns will be placed as pairs and there will be the fifth cavern between two pairs - the one with the cryogenic equipment, to provide cooling for 89K liquid argon.\\ 
Key advantages of liquid argon as a far detector working volume as described in \cite{ref_aboutLAr} are ability to act as both a target and a detector, and also to operate as a tracker and a Cherenkov detector at the same time. Liquid argon is denser than water, and therefore such detector would experience more neutrino induced reactions per unit volume than water detector would. \\



\begin{figure}
\caption{The scheme of the cross section of the LArTPC for far detector of the DUNE and far detector caverns. Sources of figures: \cite{ref_LBNFdoc_volume-detectors}}
\label{fig:farDetector_TPC}
\centering
\includegraphics[width=0.5\textwidth, keepaspectratio=true]{figs/farDetector_TPC.png}\includegraphics[width=0.5\textwidth, keepaspectratio=true]{figs/farDetector_Caverns.png}
\end{figure}

The liquid argon TPC is the main working volume of the detector. The chamber is merged into the liquid argon at tempetature of 89 K. On the figure \ref{fig:farDetector_TPC} the cathod plane assemblies (CPAs) and the anode plane assemblies (APAs) are shown. The voltages on the APAs and the CPAs are applied in such a way to create uniform electric field between anode and cathod planes. Charged particle travelling through the electron field ionizes argon atoms. Electrons induced in the ionization process drift to the APAs and produce signal on the readout electronic elements.


