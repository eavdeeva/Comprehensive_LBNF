\subsection{Near Detector}


A near detector is an important part of any long baseline neutrino oscillation experiment. It measures the primary neutrino beam flux as it is produced by the beam production system. Chapter 6 of the draft LBNF CDR \cite{ref_LBNFdoc_volume-detectors} lists the following precision measurements to be performed by the Near Detector: 
\begin{itemize}
  \item absolute flux measurement
  \item relative neutrino and antineutrino flux measurements
  \item flavor content of the neutrino source
  \item determination of the $E_\nu$-scale of neutrinos versus antineutrinos
  \item event-by-event measurements of NC interactions
  \item measurement of $\pi^0$, $\pi^\pm$, $K^\pm$, p, $K^0_S$ and $\Lambda$ in the NC and CC
\end{itemize}

More specifically, the list of the physics measurements related to the neutrino oscillations to be performed by the Near Detector includes:
\begin{itemize}
  \item fluxes of $\nu_\mu$, $\bar{\nu_\mu}$, $\nu_e$ and $\bar{\nu_e}$. To distinguish between flavors, the measurement should rely on charged current interaction (fig. \ref{fig:MuonAndNeutronDecays}, middle and right) and measure the products of these interactions $\mu^-$, $\mu^+$, $e^-$, and $e^+$. (While the beam production system has the highest probability to produce muon neutrinos, the production of certain number electron neutrinos is also possible, for example, from charged kaon decays)
  \item $\nu_e$-$\bar{\nu_e}$ assymetries. For that, it's important not only distinguish between $\mu^\pm$ and $e^\pm$ but also between $e^-$ and $e^+$.
  \item the absolute $\nu_\mu$ and $\bar{\nu_\mu}$ fluxes need to be measured with $\simeq{3\%}$ precision in the neutrino energy range 0.5-8 GeV
  \item cross section of NC versus CC processes as a function of hadronic energy. NC is one of major backrounds which contribute to neutrino oscillation measurement
  \item yields of $\pi_0$ and photons. These particles are the most significant background to $\nu_e$ and $\bar{\nu_e}$ contamination
  \item fractions of the $\pi^\pm$ into the CC and the NC hadronic jets.    
\end{itemize} 


\begin{figure}
\caption{Scheme of the DUNE Near Detector (left) and related complex (right).}
\label{fig:nearDetector}
\centering
\includegraphics[width=0.63\textwidth, keepaspectratio=true]{figs/nearDetector.png}\includegraphics[width=0.35\textwidth, keepaspectratio=true]{figs/nearDetector_project.png}
\end{figure}

The scheme of the near detector is shown at the fig. \ref{fig:nearDetector}. The detector will consist of central Straw-Tube Tracker (STT) modules, electromagnetic calorimeter (ECAL), magnet coils of 0.4T and muon identification system consisting of Resistive Plate Chamber (RPC) modules. The neutrinos would come from the bottom left corner of the picture, to the End RPCs.
The detector will be placed 60 meters underground, at least 200 m downstream of the beamline target.




