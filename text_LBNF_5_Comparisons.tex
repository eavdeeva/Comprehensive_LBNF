%\subsection{LBNF compared to the other long baseline neutrino oscillation experiments}
Table \ref{tab:compareExps} provides the comparisons of the LBNF/DUNE most important parameters with those of the other long baseline neutrino experiments: K2K, NuMI, CNGS and T2K \cite{ref_LBN_OscExpReview}. The beam power of LBNF/DUNE is planned to be 2MW while the other operating experiments have only beam powers of few hundred Watts. The baseline of LBNF/DUNE is planned to be 1300 km while the baselines of the other experiments vary from 250 km to 735 km only. The Super-Kamiokande FD has a larger mass than the LBNF/DUNE FD is going to have (50 kt vs 40 kt) but the LBNF/DUNE FD will be filled with liquid argon which is a more favorable substance for neutrino detection than water (which the Super-Kamiokande FD is filled with). The only FD which has liquid argon as working volume is ICARUS but its mass is only 0.76 kt. \\ \\
%Therefore, among the experiments discussed, LBNF/DUNE is going to have the longest baseline, the highest beam power and the FD with the highest neutrino detection efficiency. These characteristics will allow LBNF/DUNE to perform more precise measurements than previous and currently existing experiments can do and become sensitive to effects which were not observed before.
\begin{table}[h]
  \centering
  \begin{center}
  \caption{ Comparison of different long baseline neutrino oscillations experiments. Abbreviations and notations used in the table: $E_p$ - proton energy, FGD - Fine-Grained Detector, ChD - Cherenkov Detector, LAr - liquid argon }
  \begin{tabular}{|c|c|c|c|c|c|}
              & KEK (K2K) & NuMI & CNGS & T2K & LBNF/DUNE\\ \hline
     location & Japan  & Illinois - & Switzerland - & Japan & Illinois - \\ 
              &        & Minnesota & Italy &  & South Dakota\\ \hline
     accelerator & KEK PS  & FNAL & CERN's SPS & J-PARC & FNAL\\ \hline
     time of oper. & 1999-2004  & 2005-2012 & 2006-2012 & 2010- & future \\ \hline 
     beam power  &  5 kW  & 300-350 kW  & 300 kW & 750 kW & 2000 kW\\ \hline 
     $E_p$  & 12 GeV & 120 GeV & 400 GeV & 30 GeV & 60-120 GeV\\ \hline 
     baseline  & 250 km & 735 km & 730 km & 295 km & 1300 km\\ \hline 
%                & KEK (K2K)   & NuMI                & CNGS                & T2K         & LBNF (DUNE)\\ \hline
     near        & (water ChD) & MINOS               & (muon               & ND280       & DUNE (FGD)\\  
     detector(s) & (FGD)       & (track. and scint.) & detector)           & INGRID      & \\ \hline 
     ND mass     & 1 kt (ChD)  & 0.98 kt             &                     &             & \\ \hline 
     far         & SuperK      & MINOS               & ICARUS (LAr)        & SuperK      & DUNE (LAr)\\  
     detector(s) & (water ChD) & track. and scint.   & OPERA (FGD)        & (water ChD) & \\ \hline 
     FD mass     & 50 kt       & 5.4 kt              & 0.76 kt (ICARUS)   & 50 kt       & 40 kt\\ 
                 &             &                     & 1.25 kt (OPERA)    &             & \\ \hline 
 \end{tabular}
  \label{tab:compareExps}
  \end{center}
\end{table}
