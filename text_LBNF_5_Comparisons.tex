\subsection{LBNF compared to the other long baseline neutrino oscillation experiments}
The review article \cite{ref_LBN_OscExpReview} describes beams and detectors of the long baseline neutrino experiments KEK \cite{ref_KEK}, NuMI \cite{ref_NuMI}, CNGS \cite{ref_CNGS} and J-PARC \cite{ref_JPARC}. The main parameters, compared to those of the LBNF, are summarized in the table \ref{tab:compareExps}. Common in facility setups for all these experiments is that they all include neutrino beam production system incremented to large accelerator facility, tracking near detector allowing precise measurements of the initial beam parameters and large volume far detector. Japanese old experiment K2K which operated in 1999-2004 and it's update T2K use different starting poins (KEK and J-PARC) but the same far detector - Super-Kamiokande, which is 50 kilotonnes water Cherenkov detector. T2K, which already delivered many important results, including the first mearument of $\theta_{13}$, is currently operating and looking forward to perform part of the LBNF physics program too. T2K's baseline is 295 km. The experiment hosted in USA, the NuMI, as well as proposed LBNF, uses neutrino beams produced in Fermilab but it's far detector, MINOS, is located in Minnessota and the experiment's baseline is 735 km. The working volume of the MINOS is magnetized tracker and polysterene scintillator, totalling to 5.4 kilotonnes. The European experiment, the CERN Neutrinos to Gran-Sasso (CNGS), as one can tell from it's name, has it's neutrino beam produced in CERN and the neutrinos measured in Gran-Sasso, Italy. This experiment has two far detectors: fine-grained tracker OPERA and, as well as the DUNE far detector, the liquid argon time-projection chamber ICARUS. But the DUNE has much larger working volume: 4 caverns, 10 kilotonnes each, compared to 760 tonnes ICARUS. As for the beam power, the LBNF is planned to have 2MW while other operating experiments has only beam powers of few hundred Watts. Therefore, among the experiments discussed, the LBNF is going to have the longest baseline (1300 km), the highest beam power and the most sensitive detector (while Super-Kamiokande has larger volume, it's filled with water which is not as favorable for the neutrino detection as liquid argon, as discussed in the subsection "Far Detector"). These characteristics will allow the LBNF to perform more precise measurements than previous and currently existing experiments can do and become sensitive to effects which weren't observed before.

\begin{table}[h]
  \centering
  \begin{center}
  \caption{ Comparison of different long baseline neutrino oscillations experiments. Abbreviations and notations used in the table: CNGS - CERN Neutrinos to Gran-Sasso, PS - Proton Synchrotron, J-PARC - Japan Accelerator Resarch Complex, FNAL - Fermilab National Accelerator Laboratory, $E_p$ - proton energy, DUNE - Deep Underground Neutrino Experiment, FGD - Fine-Grained Detector, ChD - Cherenkov Detector, SuperK - Super-Kamiokande, MINOS - Main Injector Neutrino Oscillation Search, OPERA - Oscillation Project with Emulsion-tRacking Apparatus, ICARUS - Imaging Cosmic And Rare Underground Signals, LAr - liquid argon }
  \begin{tabular}{|c|c|c|c|c|c|}
              & KEK (K2K) & NuMI & CNGS & T2K & LBNF (DUNE)\\ \hline
     location & Japan  & Illinois - & Switzerland - & Japan & Illinois - \\ 
              &        & Minnesota & Italy &  & South Dakota\\ \hline
     accelerator & KEK PS  & FNAL & CERN's SPS & J-PARC & FNAL\\ \hline
     time of oper. & 1999-2004  & 2005-2012 & 2006-2012 & 2010- & future \\ \hline 
     beam power  &  5 kW  & 300-350 kW  & 300 kW & 750 kW & 2000 kW\\ \hline 
     $E_p$  & 12 GeV & 120 GeV & 400 GeV & 30 GeV & 60-120 GeV\\ \hline 
     baseline  & 250 km & 735 km & 730 km & 295 km & 1300 km\\ \hline 
%                & KEK (K2K)   & NuMI                & CNGS                & T2K         & LBNF (DUNE)\\ \hline
     near        & (water ChD) & MINOS               & (muon               & ND280       & DUNE (FGD)\\  
     detector(s) & (FGD)       & (track. and scint.) & detector)           & INGRID      & \\ \hline 
     ND mass     & 1 kt (ChD)  & 0.98 kt             &                     &             & \\ \hline 
     far         & SuperK      & MINOS               & ICARUS (LAr)        & SuperK      & DUNE (LAr)\\  
     detector(s) & (water ChD) & track. and scint.   & OPERA (FGD)        & (water ChD) & \\ \hline 
     FD mass     & 50 kt       & 5.4 kt              & 0.76 kt (ICARUS)   & 50 kt       & 40 kt\\ 
                 &             &                     & 1.25 kt (OPERA)    &             & \\ \hline 
 \end{tabular}
  \label{tab:compareExps}
  \end{center}
\end{table}



